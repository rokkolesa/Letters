%%%%%%%%%%%%%%%%%%%%%%%%%%%%%%%%%%%%%%%%%
% University/School Laboratory Report
% LaTeX Template
% Version 3.1 (25/3/14)
%
% This template has been downloaded from:
% http://www.LaTeXTemplates.com
%
% Original author:
% Linux and Unix Users Group at Virginia Tech Wiki 
% (https://vtluug.org/wiki/Example_LaTeX_chem_lab_report)
%
% License:
% CC BY-NC-SA 3.0 (http://creativecommons.org/licenses/by-nc-sa/3.0/)
%
%%%%%%%%%%%%%%%%%%%%%%%%%%%%%%%%%%%%%%%%%

%----------------------------------------------------------------------------------------
%	PACKAGES AND DOCUMENT CONFIGURATIONS
%----------------------------------------------------------------------------------------

\documentclass{article}

\usepackage[version=3]{mhchem} % Package for chemical equation typesetting
\usepackage{siunitx} % Provides the \SI{}{} and \si{} command for typesetting SI units
\usepackage{graphicx} % Required for the inclusion of images
\usepackage{natbib} % Required to change bibliography style to APA
\usepackage{amsmath} % Required for some math elements 
\usepackage[utf8]{inputenc} % slovenski znaki
\usepackage[slovene]{babel} % slovenski formati
\usepackage[]{algorithm2e}
\usepackage{hyperref}
\setlength\parindent{0pt} % Removes all indentation from paragraphs

\renewcommand{\labelenumi}{\alph{enumi}.} % Make numbering in the enumerate environment by letter rather than number (e.g. section 6)

%\usepackage{times} % Uncomment to use the Times New Roman font

%----------------------------------------------------------------------------------------
%	DOCUMENT INFORMATION
%----------------------------------------------------------------------------------------

\title{Letters} % Title

\author{Rok Koleša, Domen Kren, Darko Janković} % Author name

\date{\today} % Date for the report

\begin{document}

\maketitle % Insert the title, author and date

\begin{center}
\begin{tabular}{l r}
Mentor: & prof. dr. Neža Mramor Kosta % Instructor/supervisor
\end{tabular}
\end{center}

% If you wish to include an abstract, uncomment the lines below
% \begin{abstract}
% Abstract text
% \end{abstract}

\newpage
\tableofcontents
\newpage
%----------------------------------------------------------------------------------------
%	SECTION 1
%----------------------------------------------------------------------------------------

\section{Goal}
  Determine which capital letter is represented by given simplicial complex from the set of 12 letters of Slovenian alphabet. Simplicial complex is represented with a list of edges which consist of pairs of start and end points. Each point is given by a pair of coordinates.


\subsection{Definitions}
\label{definicije}
\begin{description}
\item[Filtration]
Definition...
\item[Persistent homology]
Definition...
\end{description} 
 
%----------------------------------------------------------------------------------------
%	SECTION 2
%----------------------------------------------------------------------------------------

\section{Input}
Images of letters, transformed into 1-dimensional simplicial complexes are in the files
a.png, b.png, ... , l.png
Images of the resulting simplicial complexes (so that you can see what they look like) are
in t\_a.png, t\_b.png, ... , t\_l.png
Triangulations of the letters: a.out, b.out, ... , l.out
Each file contains only edges which are given as a list of two points: the starting point
and the ending point of the edge. Each point is given by a pair of coordinates.

%----------------------------------------------------------------------------------------
%	SECTION 3
%----------------------------------------------------------------------------------------

\section{Solution}

%----------------------------------------------------------------------------------------
%	SECTION 4
%----------------------------------------------------------------------------------------

\section{Results}


%----------------------------------------------------------------------------------------
%	SECTION 5
%----------------------------------------------------------------------------------------

\section{Conclusion}



\end{document}